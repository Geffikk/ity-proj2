\documentclass[11pt, a4paper, twocolumn ]{article}

\usepackage[utf8]{inputenc}
\usepackage[top=2.5cm, left=1.5cm, text={18cm, 25cm}]{geometry}
\usepackage[czech]{babel} 
\usepackage[IL2]{fontenc}
\usepackage{amssymb}
\usepackage{amsmath}
\usepackage{amsthm}
\usepackage{turnstile}
\usepackage{caption}
\usepackage[unicode]{hyperref}
\usepackage{times}

\newtheorem{thm}{Theorem}[section]
\theoremstyle{definition}
\newtheorem{definition}{Definice}
\newtheorem{mysnt}{Věta}


\begin{document}

    \begin{titlepage}
        \begin{center}
            {\Huge \textsc{Fakulta informačních technologií \\
            \vspace{\stretch{0.015}}
                            Vysoké učení technické v Brně}}\\
            
            \vspace{\stretch{0.4}}
            \LARGE
            {Typografie a publikování\,--\,2. projekt \\
            Sazba dokumentů a matematických výrazů}\\
            \vspace{\stretch{0.6}}
    \end{center}

    {\Large {2019 \hfill Maroš Geffert (xgeffe00)}}
\end{titlepage}

\label{page}

\section*{Úvod}
V~této úloze si vyzkoušíme sazbu titulní strany, matematických vzorců, prostředí a~dalších textových struktur obvyklých pro~technicky zaměřené texty (například rovnice~(\ref{r1}) nebo Definice \ref{definition_zasobnikovy_automat} na~strane \pageref{definition_zasobnikovy_automat}). Pro odkazovaní na~vzorce a~struktury zásadně používáme příkaz \verb|\label| a~\verb|\ref| případně \verb|\pageref| pokud se chceme odkázat na~stranu výskytu.\par
Na~titulní straně je využito sázení nadpisu podle optického středu s~využitím zlatého řezu. Tento postup byl probírán na~přednášce. Dále je použito odřádkování se zadanou relativní velikostí 0.4 em a~0.3 em.

\section{Matematický text}
Nejprve se podíváme na~sázení matematických symbolů a~výrazů v~plynulém textu včetně sazby definic a~vět s~využitím balíku \texttt{amsthm}. Rovněž použijeme poznámku podčarou s~použitím příkazu \verb|\footnote|. Někdy je vhodné použít  \verb|\mbox{}|, která říká, že text nemá být zalomen.\medskip

\begin{definition}
    \label{definition_zasobnikovy_automat}
    Zásobníkový automat \emph{(ZA) je definován jako sedmice tvaru $A = (Q,\Sigma,\Gamma,\delta,q_0,Z_0,F)$, kde:}

\begin{itemize}
    \item $Q$ je \emph{konečná množina} vnitřních (řídicích) stavů,
    \item $\Sigma$ je \emph{konečná} vstupní abeceda,
    \item $\Gamma$ je \emph{konečná} zásobníková abeceda,
    \item $\delta$ je přechodová funkce $Q \times (\Sigma \cup \{{\epsilon}\}) \times \Gamma \rightarrow 2^{Q\times\Gamma^*}$,
    \item $q_0 \in Q$ je počáteční stav, $Z_0 \in \Gamma$ je startovací symbol zásobníku $F \subseteq Q$ je množina koncových stavů.
\end{itemize}
\smallskip
\par
Nechť $P = (Q, \Sigma, \Gamma, \delta, q_0, Z_0, F)$ je zásobníkový auto-mat. \emph{Konfigurací} nazveme trojici $(q, w, \alpha) \in Q \times \Sigma^{*} \times \Gamma^{*}$, kde $q$ je aktuální stav vnitřního řízení, $\omega$ je dosud~nezpracovaná část vstupního řetězce a $\alpha = Z_{i_1} Z_{i_2} \dots Z_{i_k}$ je obsah zásobníku\footnote{$Z_{i_1}$ je vrchol zásobniku}.
\end{definition} 

\subsection{Podsekce obsahující větu a odkaz}
\begin{definition}
    \label{definition_retazec}
    Řetězec $w$ nad abecedou $\Sigma$ je přijat ZA \emph{A jestliže} $(q_0, w, Z_0)\underset{A}{\overset{*}{\vdash}}(q_F, \epsilon, \gamma)$ \textit{pro nějaké} $ \gamma \in$ $\Gamma^*$ \emph{a} $q_F \in F$. 
    Množinu $L(A) = \{w\: |\: w\: \textit{je přijat ZA A}\} \subseteq \Sigma^*$ \emph{nazýváme} jazyk přijímaný TS $M$.\par \medskip
    Nyní si vyzkoušíme sazbu vět a~důkazů opět s~použitím balíku \texttt{amsthm}. \par

\end{definition}

\begin{mysnt}
    \label{snt}
    \emph {Třída jazyků, které jsou přijímány ZA, odpovídá}
    bezkontextovým jazykům.

    \begin{proof}[Důkaz]
      V~důkaze vyjdeme z~Definice \ref{definition_zasobnikovy_automat} a \ref{definition_retazec}.
    \end{proof}
\end{mysnt}

\section{Rovnice a odkazy}
Složitější matematické formulace sázíme mimo plynulý
text. Lze umístit několik výrazů na~jeden řádek, ale pak je
třeba tyto vhodně oddělit, například příkazem \verb|\quad|.\\ \\
$\sqrt[i]{x^3_i}$\quad kde $x_i$ je $i$-té sudé číslo splňující\quad $x_i^{2-{x_i^{i^2}}} \leq x_i^{y_i^3}$ \smallskip \par
V~rovnici (\ref{r1}) jsou využity tři typy závorek s~různou explicitně definovanou velikostí.

\begin{eqnarray}
    x & = & \bigg[\Big\{\big[a+b\big] * c \Big\}^d \ominus 1 \bigg]^{1/2}  \label{r1} \\
    y & = & \lim_{x\to\infty}\frac{\frac{1}{\log_{10}\,{x}}}{\sin^2\,{x} + \cos^2\,{x}}\nonumber
\end{eqnarray}\par

V~této větě vidíme, jak vypadá implicitní vysázení limity $\lim_{n \to \infty} f(n)$ v~normálním odstavci textu. Podobně je to i~s~dalšími symboly jako $\prod_{i=1}^n$ $2^i$ či $\cap_{A \in \mathcal{B}} A$. V~pří-padě vzorců $\lim\limits_{n \to \infty} f(n)$ a $\prod\limits_{i=1}^n 2^i$ jsme si vynutili méně úspornou sazbu příkazem \verb|\limits|.

\begin{eqnarray}
		\int\limits^a_b g(x) \, \mathrm{d}x & = & - \int^b_a f(x) \, \mathrm{d}x \\
		\overline{\overline{A \vee B}} & \Leftrightarrow & \overline{\overline{A} \wedge \overline{B}}
\end{eqnarray}

\section{Matice}
Pro~sázení matic se velmi často používá prostředí \texttt{array} a~závorky (\verb|\left|, \verb|\right|).

$$
\left[
\begin{array}{ccc}
 & \widehat{\beta + \gamma} & \hat{\pi} \\
\vec{a} & \overleftrightarrow{AC} &
\end{array}
\right]
= 1 \Longleftrightarrow \mathbb{Q} = \mathbf{R}
$$

$$
		\mathbf{A} =
		\left|
		\begin{array}{cccc}
			a_{11} & a_{12} & \ldots & a_{1n} \\
			a_{21} & a_{22} & \ldots & a_{2n} \\
			\vdots & \vdots & \ddots & \vdots \\
			a_{m1} & a_{m2} & \ldots & a_{mn}
		\end{array}
		\right|
		=
		\left.
		\begin{array}{cc}
			t & u \\
			v & w
		\end{array}
		\right.
		= tw - uv
$$

Prostředí \texttt{array} lze úspěšně využít i~jinde.

$$
		\binom{n}{k} =
		\left\{
		\begin{array}{ll}
		    0 & \text{pro } k < 0 \text{ nebo } k > n \\
			\frac{n!}{k! (n - k)!} & \text{pro } 0 \leq k \leq n 
		\end{array}
		\right.
$$

\end{document}

